\documentclass[english,10pt]{beamer}

\setlength{\parskip}{\smallskipamount}
\setlength{\parindent}{0pt}

\setbeamersize{text margin left=5pt, text margin right=5pt}

\usepackage{amsmath}
\usepackage{amssymb}
\usepackage{braket}

\usepackage{minted}
\newminted{julia}{breaklines,fontsize=\footnotesize,texcomments=true}
\newminted{python}{breaklines,fontsize=\footnotesize,texcomments=true}
\newminted{bash}{breaklines,fontsize=\footnotesize,texcomments=true}
\newminted{text}{breaklines,fontsize=\footnotesize,texcomments=true}

\definecolor{mintedbg}{rgb}{0.95,0.95,0.95}
\usepackage{mdframed}

\BeforeBeginEnvironment{minted}{\begin{mdframed}[backgroundcolor=mintedbg]}
\AfterEndEnvironment{minted}{\end{mdframed}}

\setcounter{secnumdepth}{3}
\setcounter{tocdepth}{3}

\makeatletter

 \newcommand\makebeamertitle{\frame{\maketitle}}%
 % (ERT) argument for the TOC
 \AtBeginDocument{%
   \let\origtableofcontents=\tableofcontents
   \def\tableofcontents{\@ifnextchar[{\origtableofcontents}{\gobbletableofcontents}}
   \def\gobbletableofcontents#1{\origtableofcontents}
 }

\makeatother

\usepackage{babel}

\begin{document}


\title{\texttt{PWDFT.jl}: Density Functional Theory Calculations with Julia}
\author{Fadjar Fathurrahman}
\institute{
Engineering Physics Department \\
Research Center for Nanoscience and Nanotechnology \\
Institut Teknologi Bandung
}
\date{20 April 2018}

\frame{\titlepage}

\begin{frame}
\frametitle{Codes for DFT}

\begin{itemize}
\item using Fortran, C, C++: ABINIT, VASP, Quantum Espresso
\item Python: GPAW
\end{itemize}

\end{frame}


\begin{frame}
\frametitle{PWDFT.jl}

\begin{itemize}
\item A package for doing DFT calculations using Julia
\item plane wave basis + pseudopotentials (norm-conserving)
\item LDA and GGA functionals can be used (via LibXC)
\item symmetry detection (via SPGLIB)
\item algorithms: SCF and Emin (PCG and DCM)
\end{itemize}

\end{frame}


\begin{frame}
\frametitle{Aim}

\begin{itemize}
\item Friendly-to-develop DFT package
\item quick development for testing various algorithms
\end{itemize}

\end{frame}


\begin{frame}[fragile]
\frametitle{Example}

\begin{juliacode}
using PWDFT
atoms = init_atoms_xyz("H2.xyz")
pspfiles = ["../pseudopotentials/pade_gth/H-q1.gth"]
LatVecs = 16.0*diagm( ones(3) )
ecutwfc = 15.0 # in Hartree
Ham = PWHamiltonian( atoms, pspfiles, ecutwfc, LatVecs )
Ham.energies.NN = calc_E_NN( Ham.pw, atoms, [1.0] )
KS_solve_SCF!( Ham, update_psi="CheFSI" )
\end{juliacode}

\end{frame}

\end{document}

