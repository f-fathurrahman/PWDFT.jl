\documentclass[a4paper,10pt]{paper}

\input{PREAMBLE}

\usepackage[a4paper]{geometry}
%\geometry{verbose,tmargin=1.5cm,bmargin=1.5cm,lmargin=1cm,rmargin=2cm}

\setlength{\parskip}{\smallskipamount}
\setlength{\parindent}{0pt}

\usepackage{hyperref}
\usepackage{url}
\usepackage{xcolor}

\usepackage{amsmath}
\usepackage{amssymb}
\usepackage{braket}

\newenvironment{markdown}%
    {\VerbatimEnvironment\begin{VerbatimOut}{tmp.markdown}}%
    {\end{VerbatimOut}%
        \immediate\write18{pandoc tmp.markdown -t latex -o tmp.tex}%
        All atomic systems are assumed to be periodic.

The definition of type \texttt{Atoms} is given below.

\begin{Shaded}
\begin{Highlighting}[]
\NormalTok{mutable struct Atoms}
\NormalTok{    Natoms::}\DataTypeTok{Int64}
\NormalTok{    Nspecies::}\DataTypeTok{Int64}
\NormalTok{    positions::}\DataTypeTok{Array}\NormalTok{\{}\DataTypeTok{Float64}\NormalTok{,}\FloatTok{2}\NormalTok{\}}
\NormalTok{    atm2species::}\DataTypeTok{Array}\NormalTok{\{}\DataTypeTok{Int64}\NormalTok{,}\FloatTok{1}\NormalTok{\}}
\NormalTok{    atsymbs::}\DataTypeTok{Array}\NormalTok{\{}\DataTypeTok{String}\NormalTok{,}\FloatTok{1}\NormalTok{\}}
\NormalTok{    SpeciesSymbols::}\DataTypeTok{Array}\NormalTok{\{}\DataTypeTok{String}\NormalTok{,}\FloatTok{1}\NormalTok{\}}
\NormalTok{    LatVecs::}\DataTypeTok{Array}\NormalTok{\{}\DataTypeTok{Float64}\NormalTok{,}\FloatTok{2}\NormalTok{\}}
\NormalTok{    Zvals::}\DataTypeTok{Array}\NormalTok{\{}\DataTypeTok{Float64}\NormalTok{,}\FloatTok{1}\NormalTok{\}}
\KeywordTok{end}
\end{Highlighting}
\end{Shaded}

Information about \texttt{LatVecs} and \texttt{Zvals} are also available
from \texttt{PWGrid} and \texttt{PsPots}. They are included to reduce
number of required arguments to several functions.

Currently, the following functions are provided to initialize an
\texttt{Atoms}:

\begin{Shaded}
\begin{Highlighting}[]
\NormalTok{atoms = Atoms() }\CommentTok{# dummy constructor}
\NormalTok{atoms = init_atoms_xyz(filexyz; in_bohr=false, verbose=false)}
\NormalTok{atoms = init_atoms_xyz_string(filexyz; in_bohr=false, verbose=false)}
\end{Highlighting}
\end{Shaded}

Note that, \texttt{LatVecs} must be set manually by:

\begin{Shaded}
\begin{Highlighting}[]
\NormalTok{atoms.LatVecs = }\FloatTok{16}\NormalTok{*eye(}\FloatTok{3}\NormalTok{) }\CommentTok{# for example}
\end{Highlighting}
\end{Shaded}

\texttt{Zvals} is set when constructing \texttt{PWHamiltonian}.

Equation \begin{equation}
\frac{\alpha}{\beta}
\end{equation}
}

\begin{document}

\title{\texttt{PWDFT.jl} Developer's Note}
\author{Fadjar Fathurrahman}
\maketitle

\begin{markdown}
`PWDFT.jl` uses Hartree atomic units.
\end{markdown}

\section{Status}

\begin{markdown}

- **28 May 2018** The following features are working now:
	
  - LDA and GGA, spin-paired and spin polarized calculations
  - Calculation with k-points (for periodic solids). Monkhorst-Pack grid
    generation is done using \texttt{spglib}
  
  Band structure calculation is possible in principle as simply solving
  Schrodinger equation with converged Kohn-Sham potentials, however there
  is currently no tidy script or function to do that.

  Total energy result for isolated systems (atoms and molecules) agrees quite
  well with ABINIT and PWSCF results.

  Total energy result for periodic solid is quite different from ABINIT and PWSCF.
  I suspect that this is related to treatment of electrostatic terms in periodic system.

  SCF is quite shaky for several systems, however it is working in quite well in nonmetallic
  system.

\end{markdown}


\section{Describing an atomic system}

\begin{markdown}
Because plane wave basis set is used, all atomic systems are assumed
to be periodic. For isolated molecular systems, a periodic bounding box
must be specified.

Currently, the definition of type `Atoms` is given below.

```julia
mutable struct Atoms
    Natoms::Int64
    Nspecies::Int64
    positions::Array{Float64,2}
    atm2species::Array{Int64,1}
    atsymbs::Array{String,1}
    SpeciesSymbols::Array{String,1}
    LatVecs::Array{Float64,2}
    Zvals::Array{Float64,1}
end
```

Information about `LatVecs` and `Zvals` are also available
from `PWGrid` and `PsPotGTH`, respectively.
They are included to reduce number of required arguments to several functions.

`LatVecs` represents lattice vectors $a_{1}$, $a_{2}$, dan $a_{3}$, stacked by column
in $3\times3$ matrix.

`Zvals` is set when constructing `PWHamiltonian`. The default value is `zeros(Nspecies)`.

Currently, the following functions are provided to initialize an `Atoms`:

- Using dummy constructor:
  ```julia
  atoms = Atoms()
  ```
  It is important to set other fields of `atoms` accordingly.

- Using xyz-like structure
  ```julia
  atoms = init_atoms_xyz(filexyz; in_bohr=false, verbose=false)
  atoms = init_atoms_xyz_string(filexyz; in_bohr=false, verbose=false)
  ```
  In the first function, `filexyz` is a string representing path of the xyz file while
  in the second function `filexyz` represent directly the content of xyz file.
  Example:
  ```julia
  # Initialize using an existing xyz file
  atoms = init_atoms_xyz("H2O.xyz")
  # Initialize using 'inline' xyz file
  atoms = init_atoms_xyz(
  """
  2

  O   -0.8  0.0  0.0
  O    0.8  0.0  0.0
  """
  ```

Note that, for both ways `LatVecs` must be set manually by:
```julia
atoms.LatVecs = 16*eye(3) # for example
```
Currently there is no warning or check being performed to make sure that `LatVecs`
is defined properly. The default value is `zeros(3,3)`. So an error will happen
if an instance of `PWGrid` is constucted because we will try to invert a zero matrix.


Equation
\begin{equation}
\frac{\alpha}{\beta}
\end{equation}


\end{markdown}



\section{Hamiltonian}

\begin{markdown}
`Hamiltonian` is the central object.
It designed such that we can perform application or multiplication of
Hamiltonian to wave function:

```julia
Hpsi = op_H(H, psi)
```
or (by 'overloading' `*` )
```julia
Hpsi = H*psi
```
\end{markdown}



\section{k-points (Bloch wave vectors)}

Monkhorst-Pack grid (used for Brillouin-zone integration)
\begin{equation}
\sum_{i=1,2,3} \frac{2n_i -N_i - 1}{2N_i} \mathbf{b}_i
\end{equation}

where $n_i=1,2,...,N_{i}$
size = \((N_1, N_2, N_3)\) and the \(\mathbf{b}_i\)'s are reciprocal lattice vectors.

Please see file \texttt{kpoint\_grid.f90} in \verb|PW/src|.

\end{document}
