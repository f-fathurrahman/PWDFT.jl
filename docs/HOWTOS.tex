\subsection{Referring or including files in \jlinline{sandbox} (or other dirs in \jlinline{PWDFT.jl})}

\begin{juliacode}
using PWDFT
const DIR_PWDFT = joinpath(dirname(pathof(PWDFT)),"..")
const DIR_PSP = joinpath(DIR_PWDFT,"pseudopotentials","pade_gth")
const DIR_STRUCTURES = joinpath(DIR_PWDFT, "structures")

pspfiles = [joinpath(DIR_PSP,"Ag-q11.gth")]
\end{juliacode}


\subsection{Using Babel to generate xyz file from SMILES}

\begin{textcode}
babel file.smi file.sdf
babel file.sdf file.xyz
\end{textcode}

Use \txtinline{babel -h} to autogenerate hydrogens.



\subsection{Setting up pseudopotentials}

One can use the function \jlinline{get_default_psp(::Atoms)} to get default
pseudopotentials set for a given instance of \jlinline{Atoms}.

Currently, it is not part of main \jlinline{PWDFT.jl} package. It is located
under \txtinline{sandbox} subdirectory of \jlinline{PWDFT.jl} distribution.

\begin{juliacode}
using PWDFT

DIR_PWDFT = jointpath(dirname(pathof(PWDFT)),"..")
include(jointpath(DIRPWDFT,"sandbox","get_default_psp.jl"))

atoms = Atoms(ext_xyz_file="atoms.xyz")
pspfiles = get_default_psp(atoms)
\end{juliacode}

Alternatively, one can set \txtinline{pspfiles} manually because it is simply
an array of \jlinline{String}:
\begin{juliacode}
pspfiles = ["Al-q3.gth", "O-q6.gth"]
\end{juliacode}

\textbf{IMPORTANT} Be careful to set the order of species to be same as
\jlinline{atoms.SpeciesSymbols}. For example, if
\begin{juliacode}
atoms.SpeciesSymbols = ["Al", "O", "H"]
\end{juliacode}
then
\begin{juliacode}
pspfiles = ["Al-q3.gth", "O-q6.gth", "H-q1.gth"]
\end{juliacode}

\subsection{Initializing Hamiltonian}

For molecular systems:
\begin{juliacode}
Ham = Hamiltonian( atoms, pspfiles, ecutwfc )
\end{juliacode}

For insulator and semiconductor solids:
\begin{juliacode}
Ham = Hamiltonian( atoms, pspfiles, ecutwfc, meshk=[3,3,3] )
\end{juliacode}

For metallic systems:
\begin{juliacode}
Ham = Hamiltonian( atoms, pspfiles, ecutwfc, meshk=[3,3,3], extra_states=4 )
\end{juliacode}

Empty extra states can be specified by using \jlinline{extra_states} keyword.

For spin-polarized systems, \jlinline{Nspin} keyword can be used.


\subsection{Iterative diagonalization of Hamiltonian}

\begin{juliacode}
evals =  diag_LOBPCG!( Ham, psiks, verbose=false, verbose_last=false,
                       Nstates_conv=Nstates_occ )
\end{juliacode}


\subsection{Calculating electron density}

Several ways:
\begin{juliacode}
Rhoe = calc_rhoe( Nelectrons, pw, Focc, psiks, Nspin )
# or
Rhoe = calc_rhoe( Ham, psiks )
# or
calc_rhoe!( Ham, psiks, Rhoe )
\end{juliacode}

\subsection{Read and write array (binary file)}

Write to binary files:
\begin{juliacode}
for ikspin = 1:Nkpt*Nspin
    wfc_file = open("WFC_ikspin_"*string(ikspin)*".data","w")
    write( wfc_file, psiks[ikspin] )
    close( wfc_file )
end
\end{juliacode}

Read from binary files:
\begin{juliacode}
psiks = BlochWavefunc(undef,Nkpt)
for ispin = 1:Nspin, ik = 1:Nkpt
    ikspin = ik + (ispin-1)*Nkpt
    # Don't forget to use read mode
    wfc_file = open("WFC_ikspin_"*string(ikspin)*".data","r")
    psiks[ikspin] = Array{ComplexF64}(undef,Ngw[ik],Nstates)
    psiks[ikspin] = read!( wfc_file, psiks[ikspin] )
    close( wfc_file )
end
\end{juliacode}




\subsubsection*{Subspace rotation}

In case need sorting:
\begin{juliacode}
Hr = psiks[ikspin]' * op_H( Ham, psiks[ikspin] )
evals, evecs = eigen(Hr)
evals = real(evals[:])

# Sort in ascending order based on evals
idx_sorted = sortperm(evals)

# Copy to Hamiltonian
Ham.electrons.ebands[:,ikspin] = evals[idx_sorted]

# and rotate
psiks[ikspin] = psiks[ikspin]*evecs[:,idx_sorted]
\end{juliacode}

Usually we don't need to sort the eigenvalues if we use Hermitian matrix. We can calculate the
subspace Hamiltonian by:
\begin{juliacode}
evals, evecs = eigen(Hermitian(Hr))
\end{juliacode}
