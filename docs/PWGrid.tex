\section{Plane wave basis set, real space grid, and k-points}

The type \jlinline{PWGrid} wraps various variables related to plane wave basis
set. This has two fields of type \jlinline{GVectors}
and \jlinline{GVectorsW} for storing information about $\mathbf{G}$-vectors
that are used in potential and wave functions, respectively.

\begin{figure}[H]
\centering
\begin{tikzpicture}
\node[mybox] (box) {%
\begin{minipage}{0.6\textwidth}%
\begin{minted}{julia}
struct PWGrid
    ecutwfc::Float64
    ecutrho::Float64
    Ns::Tuple{Int64,Int64,Int64}
    LatVecs::Array{Float64,2}
    RecVecs::Array{Float64,2}
    CellVolume::Float64
    r::Array{Float64,2}
    gvec::GVectors
    gvecw::GVectorsW
    planfw
    planbw
end
\end{minted}
\end{minipage}
};
\node[fancytitle, right=10pt] at (box.north west) {\jlinline{PWGrid} struct definition};
\end{tikzpicture}
\caption{Definition of \jlinline{PWGrid}. The type annotation of \jlinline{planfw} and \jlinline{planbw} is
omitted because they are too long.}
\end{figure}

The $\mathbf{G}$-Gvectors can be defined as:
\begin{equation}
\mathbf{G} = n_1 \mathbf{b}_1 + n_2 \mathbf{b}_2 + n_3 \mathbf{b}_3
\end{equation}
where $n_1, n_2, n_3$ are integer numbers and
$\mathbf{b}_1, \mathbf{b}_2, \mathbf{b}_3$ are three vectors describing
unit cell of reciprocal lattice or \textit{unit reciprocal lattice vectors}.

\begin{equation}
\mathbf{b}_1 = 2\pi\frac{\mathbf{a}_{2} \times \mathbf{a}_{3}}{\Omega}
\end{equation}

A periodic function
\begin{equation}
f(\mathbf{r}) = f(\mathbf{r}+\mathbf{L}),\,\,\,
\mathbf{L} = n_{1}a_{1} + n_{2}a_{2} + n_{3}a_{3}
\end{equation}
can be expanded using plane wave basis basis functions as:
\begin{equation}
f(\mathbf{r}) = \frac{1}{\sqrt{\Omega}}\sum_{\mathbf{G}}
C_{\mathbf{G}} \exp(\imath \mathbf{G} \cdot \mathbf{r})
\end{equation}
where $C_{\mathbf{G}}$ are expansion coefficients. This sum is usually truncated
at a certain maximum value of $\mathbf{G}$-vector, $\mathbf{G}_{\mathrm{max}}$.

Kohn-Sham wave function:
\begin{equation}
\psi_{i,\mathbf{k}}(\mathbf{r}) = u_{i,\mathbf{k}}(\mathbf{r}) \exp\left[ \imath \mathbf{k} \cdot \mathbf{r} \right]
\end{equation}
where $u_{i,\mathbf{k}}(\mathbf{r}) = u_{i,\mathbf{k}}(\mathbf{r}+\mathbf{L})$

Using plane wave expansion:
\begin{equation}
u_{i,\mathbf{k}}(\mathbf{r}) =
\frac{1}{\sqrt{\Omega}}\sum_{\mathbf{G}} C_{i,\mathbf{k},\mathbf{G}} \exp(\imath \mathbf{G} \cdot \mathbf{r}),
\end{equation}
%
we have:
\begin{equation}
\psi_{i,\mathbf{k}}(\mathbf{r}) =
\frac{1}{\sqrt{\Omega}}\sum_{\mathbf{G}} C_{i,\mathbf{G+\mathbf{k}}}
\exp\left[ \imath (\mathbf{G}+\mathbf{k}) \cdot \mathbf{r} \right]
\end{equation}



An instance of \jlinline{PWGrid} can be initialized by using its constructor
which has the following signature:
\begin{juliacode}
function PWGrid( ecutwfc::Float64, LatVecs::Array{Float64,2};
    kpoints=nothing, Ns_=(0,0,0) )
\end{juliacode}
There are two mandatory arguments: \jlinline{ecutwfc} and \jlinline{LatVecs}.
\jlinline{ecutwf} is cutoff energy for kinetic energy (in Hartree) and
\jlinline{LatVecs} is usually correspond to the one used in an
instance of \jlinline{Atoms}.

Real space grid points:
$$
\mathbf{r} = \frac{i}{N_{s1}}\mathbf{a}_{1} + \frac{j}{N_{s2}}\mathbf{a}_{2} +
\frac{k}{N_{s3}}\mathbf{a}_{3}
$$

$i = 0,1,\ldots,N_{s1}-1$

$j = 0,1,\ldots,N_{s2}-1$

$k = 0,1,\ldots,N_{s3}-1$

FFT


operators op nabla op nabla 2
