\section{Atomic structure}
%
The type \jlinline{Atoms} contains the following information:
%
\begin{itemize}
\item Number of atoms: \jlinline{Natoms::Int64}
\item Number of atomic species: \jlinline{Nspecies::Int64}
\item Atomic coordinates: \jlinline{positions::Array{Float64,2}}
\item Unit cell vectors (lattice vectors): \jlinline{LatVecs::Array{Float64,2}}
\end{itemize}
%
\jlinline{Atoms} also contains several other fields such as \jlinline{Zvals}
which will be set according to the pseudopotentials assigned to
the instance of \jlinline{Atoms}.
\footnote{Not all fields of \jlinline{Atoms} (or any custom types defined in \textsf{PWDFT.jl})
are listed. The most up to date definition can be consulted in the corresping source code.}

\begin{figure}[H]
\centering
\begin{tikzpicture}
\node[mybox] (box) {%
\begin{minipage}{0.6\textwidth}%
\begin{minted}[fontsize=\footnotesize]{julia}
mutable struct Atoms
  Natoms::Int64
  Nspecies::Int64
  positions::Array{Float64,2}
  atm2species::Array{Int64,1}
  atsymbs::Array{String,1}
  SpeciesSymbols::Array{String,1}
  LatVecs::Array{Float64,2}
  Zvals::Array{Float64,1}
end
\end{minted}
\end{minipage}
};
\node[fancytitle, right=10pt] at (box.north west) {\jlinline{Atoms} struct definition};
\end{tikzpicture}
\caption{Definition of \jlinline{Atoms} struct which contains variables needed to describe
crystalline of molecular structure.}
\end{figure}

\jlinline{LatVecs} is a $3\times3$ matrix. The vectors are stored column-wise which is
opposite to the PWSCF input convention.
Several convenience functions to generate lattice vectors for Bravais lattices
are provided in \textsf{PWDFT.jl}. These functions adopt PWSCF definition. Some examples
are listed below.
\begin{itemize}
\item \jlinline{gen_lattice_sc} or \jlinline{gen_lattice_cubic} for generating
simple cubic lattice vectors.
\item \jlinline{gen_lattice_fcc}: for fcc structure
\item \jlinline{gen_lattice_bcc}: for bcc structure
\item \jlinline{gen_lattice_hcp}: for hcp structure
\end{itemize}
Please see file \txtinline{gen_lattice.jl} for more information.


There are several ways to initialize an instance of \jlinline{Atoms}. The following
are typical cases.
%
\begin{itemize}
%
\item From xyz file. We need to supply the path to xyz file as string and
set the lattice vectors:
%
\begin{juliacode}
atoms = Atoms(xyz_file="file.xyz", LatVecs=gen_lattice_sc(16.0))
\end{juliacode}
%
\item For crystalline systems, using keyword argument \jlinline{xyz_string_frac}
is sometimes convenient:
%
\begin{juliacode}
atoms = Atoms(xyz_string_frac=
        """
        2

        Si  0.0  0.0  0.0
        Si  0.25  0.25  0.25
        """, in_bohr=true,
        LatVecs=gen_lattice_fcc(10.2631))
\end{juliacode}
%
\textbf{IMPORTANT} We need to be careful to also specify \jlinline{in_bohr} keyword to get
the correct coordinates in bohr (which is used internally in \jlinline{PWDFT.jl}).
%
\item From extended xyz file, the lattice vectors information is included
along with several others information, if any, however they are ignored):
%
\begin{juliacode}
atoms = Atoms(ext_xyz_file="file.xyz")
\end{juliacode}
%
\end{itemize}
